%!TEX root = ../thesis.tex
%*******************************************************************************
%****************************** Third Chapter **********************************
%*******************************************************************************
\chapter{Método}

% **************************** Define Graphics Path **************************
\ifpdf
    \graphicspath{{Chapter3/Figs/Raster/}{Chapter3/Figs/PDF/}{Chapter3/Figs/}}
\else
    \graphicspath{{Chapter3/Figs/Vector/}{Chapter3/Figs/}}
\fi

\section{Estudio de tecnologías}

\subsection{Proceso de aprendizaje}

Para definir el proceso de aprendizaje llevado a cabo pienso que es necesario hacer una división en
lo que claramente define mi conocimiento en el antes de comenzar este informe y lo que es el
durante y el después. Durante el desarrollo del informe y la aplicación se pueden separar tres
grandes conceptos: Node.js, Blockchain y Ethereum.

Por un lado, Node.js es una tecnología con la que trabajo a diario hace más de dos años y por este
motivo no será el centro de atención a lo largo del informe. Si bien los contratos que se escriban, 
testeen y deployen lo harán en principio sobre una plataforma escrita en Node.js, se le dará suma
importancia a lo incurrido en los otros dos conceptos.

Por otro lado, Blockchain y Ethereum son dos conceptos completamente ajenos a mi experiencia. Si 
bien se oyen a diario cada vez más seguido, jamás antes había siquiera leído un informe o un libro
sobre ninguno de estos dos conceptos. Este hecho de escucharlos y leerlos todos los días en 
noticias de tecnología me hizo pensar en profundizar en ellos para conocer su funcionamiento 
interno, sus posibles aplicaciones y lograr junto con este informe, crear y deployar contratos 
inteligentes en las redes de pruebas, teniendo pleno conocimiento de lo que está sucediendo en el 
detrás de escena y habiendo intervenido todo lo que en mi responsabilidad quedaba hacerlo.

Dado que a lo largo de la facultad no se ha visto contenido alguno relacionado a Blockchain y 
Ethereum, todo lo aprendido en este proceso fue adquirido de libros, artículos de blogs 
técnicos, los propios papers oficiales de Bitcoin y Ethereum, videos de conferencias, videos 
académicos, informes y otras fuentes de menor importancia.

El proceso de aprendizaje comienza mediados del mes de Noviembre de 2018 donde se comienza a 
recolectar todos los recursos "en crudo" considerados importantes para la finalidad del informe.
Es decir, libros, artículos, informes, papers, cursos, conferencias, etc. Los recursos aquí 
buscados fueron, en números, un 85\% sobre Blockchain y Ethereum y el 15\% restante sobre como
comunicar la plataforma de Node.js con los contratos inteligentes y los nodos de las redes de 
prueba de Ethereum.

Ya comenzado el mes de Diciembre, se procedió a filtrar todos los recursos en crudo para dejar
lo considerado más importante y que me iba a dar más conocimiento de los temas para llevar adelante
el informe y el proyecto.

Rondando mediados de Diciembre y luego de haber incurrido en algunos libros, informes y videos 
académicos, se está en condiciones de comenzar a escribir una aplicación combinando Node.js, 
Ethereum y Blockchain. A partir de este momento el progreso realizado se verá reflejado luego 
en la sección "Estado alcanzado" más adelante.


\subsection{Estado alcanzado}

\section{Proceso de desarrollo}


\subsection{Puesta en marcha del entorno de trabajo}

Para llevar adelante tanto el presente informe como la aplicación, se utilizó una única estación
de trabajo descripta a continuación:

\begin{itemize}
\item Modelo: Ultrabook Dell Latitude E7270
\item Memoria RAM: 8GB DDR4 2133mhz.
\item CPU: Intel Core i7 6600U @ 2.60GHz.
\item Almacenamiento: Disco SSD LITEON L8H-256V2G-11 M.2 2280 256GB.
\item Sistema Operativo: Debian GNU/Linux 9.4 (stretch) 64 bits.
\end{itemize}

Enumeraré a continuación lo que se debió instalar en dicha estación de trabajo. No daré
practicamente detalles debido a que de muchos puntos se ha hablado en otras secciones:

\begin{itemize}
\item LaTeX https://packages.debian.org/stretch/texlive-full
\item Texmaker https://packages.debian.org/jessie/tex/texmaker
\item nvm (Node Version Manager) https://github.com/creationix/nvm
\item Node.js v10 LTS https://nodejs.org/
\item Módulos externos Node.js:
	\begin{itemize}
		\item Mocha https://github.com/mochajs/mocha
	\end{itemize}
\item MetaMask https://metamask.io/
\item 
\item
\item 
\item 
\item 
\item 
\item 
\item 
\item 
\end{itemize}

\subsection{Sprint a...}

\subsection{...Sprint n}