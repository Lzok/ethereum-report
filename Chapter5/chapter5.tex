%!TEX root = ../thesis.tex
%*******************************************************************************
%****************************** Fifth Chapter *********************************
%*******************************************************************************

\chapter{Conclusión}

\ifpdf
    \graphicspath{{Chapter5/Figs/}{Chapter5/Figs/PDF/}{Chapter5/Figs/}}
\else
    \graphicspath{{Chapter5/Figs/}{Chapter5/Figs/}}
\fi


\section{Resultados}
Se completó el desarrollo en un tiempo aceptable y cumpliendo los objetivos personales propuestos
tanto en la aplicación desarrollada como en el presente informe. Se incorporaron nuevas tecnologías
como blockchain, Ethereum y Solidity. También se profundizaron otras como Node y React, esta última
particularmente.

Gracias al hincapié de estudiar e investigar las nuevas tecnologías elegidas para el proyecto, se
comenzó a trabajar en el código con una cierta confianza conociendo de manera sólida todos los
procesos que estaban pasando detrás de la implementación del código.

Considero un éxito el presente desarrollo debido a la gran cantidad de nuevo conocimiento
incorporado en tecnologías que se encuentran en auge y por el hecho de haberlo plasmado en una app
100\% funcional.


\subsection{Posibilidades con este informe como punto de partida}
En la presente sección se enumerarán algunas opciones viables en caso de que alguien decida tomar
como punto de partida este informe para su tesina. Un punto importante es que destaco que las
opciones presentadas son altamente recomendadas para los alumnos que no posean experiencia laboral 
o que recién hayan comenzado su carrera profesional.

\begin{itemize}
	\item Actualizar Node y todas las dependencias utilizadas a su última versión estable. Esto
	obligará también a actualizar el código del contrato Solidity y los scripts de compilación y
	deploy. Actualizar dependencias y código acorde es una práctica común en el ámbito laboral real
	y muchas veces resulta ser excesivamente complejo llevarlo a cabo por todas las consideraciones
	que	se deben tener en cuenta respecto al código en producción. 
	
	\item Desarrollar dentro de la app proyectada un sistema de usuarios, donde para acceder a la
	app el usuario deba registrarse y loguearse (en favor de esto y de las prácticas comunes en el
	mercado laboral será agregar social login). A cada usuario le serán añadidas sus wallets y al
	momento de transaccionar se deberá asegurar que el wallet logueado en Metamask coincida con
	alguno de los adjuntos al usuario. A su vez, relacionar también los proyectos con los usuarios,
	donde los proyectos tendrán un usuario como owner/responsable (al igual que el contrato ahora
	mismo) y los usuarios tendrán un registro de los proyectos que participan como backers y los
	payments que aprueban o desaprueban. Queda a elección si usar una base de datos SQL o NoSQL.
	
	\item Desarrollar "tiers" de pagos y recompensas. Además de establecer una contribución mínima,
	establecer también escalones de contribución, es decir, diferentes rangos de ether con los
	cuales los usuarios puedan contribuir y en cada rango establecer diferentes recompensas
	para dar a los usuarios si el proyecto tiene éxito.	
	
	\item Investigar y desarrollar desde el punto de vista de la inversión en criptomonedas debido
	a que en el presente informe se ha indicado que no se hablaría en absoluto sobre inversiones.

\end{itemize}
