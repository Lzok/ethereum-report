%!TEX root = ../thesis.tex
%*******************************************************************************
%****************************** Second Chapter *********************************
%*******************************************************************************

\chapter{Marco teórico}

\ifpdf
    \graphicspath{{Chapter2/Figs/}{Chapter2/Figs/PDF/}{Chapter2/Figs/}}
\else
    \graphicspath{{Chapter2/Figs/}{Chapter2/Figs/}}
\fi


\section{Blockchain}
Blockchain o cadena de bloques es un registro público de transacciones que se
mantiene mediante una red distribuida de computadores, que no requiere respaldo de
ninguna autoridad central o una tercera parte y que ofrece un esquema transaccional
libre de intermediarios, gracias al uso de algoritmos criptográficos. Puede compararse con el libro
de registros de contabilidad de una empresa en donde se registran todas las entradas
y salidas de dinero, aunque en este caso hablamos de un libro de acontecimientos
digitales que no requiere de un intermediario centralizado que identifique y certifique la
información, sino que está distribuida en múltiples nodos independientes entre sí que
la registran y la validan sin necesidad de que haya confianza entre ellos.

Esta cadena de bloques (blockchain) consta de tres componentes fundamentales: transacciones,
registro y un sistema que las verifica y almacena en bloques. Cada bloque se genera a través de un
software que registra cronológicamente la información sobre cuándo y en qué secuencia han tenido
lugar las transacciones, de allí deriva su nombre. 
Esta tecnología permite que se realicen las transferencias electrónicas de una manera segura sin la
presencia de un tercero de confianza, dando solución a la principal barrera técnica de las últimas
décadas para los desarrolladores tecnológicos, el problema del doble gasto.

Por fuera de Blockchain, las transferencias electrónicas requieren intermediarios financieros para
dar confianza y seguridad a la transacción. Los intermediarios financieros generan confianza y
seguridad, preservando un registro único y centralizado de las operaciones electrónicas que permite
controlar los saldos de los titulares de cuentas y, en última instancia, garantizar la autenticidad
de una transacción. Sin intermediarios, las unidades de valor electrónicas -pesos o dólares- pueden
ser copiados y usados dos veces, tal como cualquier documento digital puede ser copiado
indefinidamente. En Blockchain, una vez introducida la información, no puede ser borrada o
modificada, solo se podrán añadir nuevos registros y no será legitimada a menos que la mayoría de
los actores se pongan de acuerdo para hacerlo.

Se podría entender esta tecnología con fundamento en sus tres características generales de la
siguiente forma: i) Blockchain es una tecnología "sin confianza", lo que significa que por primera
vez en la historia, intercambios de valor a través de una red de computadores pueden ser
verificados, monitoreados y asegurados sin la presencia de un tercero de confianza o
de una institución central; ii) es una tecnología de autenticación y verificación, que permite de
forma más eficiente las transferencias de títulos y la verificación de propiedad y iii) por ser una
tecnología sin fronteras y sin fricción, puede proporcionar una más económica y rápida
infraestructura para el intercambio de unidades de valor.

Esta tecnología surgió como el soporte tecnológico del Bitcoin y luego fue adoptada por múltiples
sectores en donde existía la necesidad de un registro o de intercambio de valor, como se verá más
adelante. Para comprender el funcionamiento general del Blockchain, se explicará la primera versión
que sirvió de base para los desarrollos en los que se está trabajando en la actualidad.


\subsection{¿Cómo funciona?}
Veremos más sobre el funcionamiento de Ethereum en otros capítulos. Aquí se intentará explicar de 
manera generalizada el funcionamiento de una blockchain.

Para generar un contexto, supondremos que Alpha quiere transferir a Beta una determinada
cantidad de unidades de valor (Alguna criptomoneda, pesos, dólares, etc.) y que ambos tienen acceso
a una billetera o un monedero en el celular, un computador o una web que les permite enviar o
recibir la moneda. Cuando Alpha decide gastar sus unidades de valor, lo que realmente está haciendo
es enviar una instrucción de cambio a la base de datos informando que parte de sus unidades de
valor ahora pertenecen a Beta. Esta instrucción es difundida en la red verificando que Alpha tiene
recursos para pagar y, si todo se encuentra correcto, se compila con otras transacciones
en un bloque con información relativa a los últimos diez minutos.
Este bloque mezcla la información de las direcciones de las partes involucradas en cada
transacción, la cantidad de unidades de valor en movimiento y una marca de tiempo, y luego las
procesa a través de una función llamada Hash. Esta función es un algoritmo criptográfico, que se
encarga de condensar en un único digito de 64 caracteres información de cualquier extensión.
Por ejemplo, el hash (encriptado con el algoritmo SHA-256) para la palabra “blockchain” sería
EF7797E13D3A75526946A3BCF00DAEC9FC9C9C4D51DDC7CC5DF888F74DD434D1 y el de “Blockchain”
625DA44E4EAF58D61CF048D168AA6F5E492DEA166D8BB54EC06C30DE07DB57E1. Preste especial atención que ante
cambios como la capitalización de una letra o la longitud del texto, el hash es completamente
diferente.

Este hash se combina con la solución-hash del bloque anterior, y se convierte en el encabezado del
bloque nuevo que se encuentra en validación. A su vez este es la base de un problema matemático que
se resuelve usando de nuevo la función Hash. La respuesta a este acertijo es solucionado por la red
en un proceso de prueba y error. Cuando finalmente algún nodo de la red encuentra la solución, esta
es compartida con el resto para su validación, en un proceso llamado “proof of work” (prueba de
trabajo). Después que ha sido aprobada por la mayoría de la red, el bloque es añadido a la cadena y 
con ello todas las transacciones contenidas en él, incluido el pago de Alpha hacia Beta.

La blockchain establece la confianza entre dos partes en una transacción a través de un libro de 
contabilidad público descentralizado y un mecanismo criptográfico que garantiza que las 
transacciones no pueden cambiarse después de materializadas.

\section{Ethereum}
Ethereum aparece a finales del año 2013 en un paper publicado por Vitalik Buterin con el objetivo
de proveer un sistema descentralizado capaz de correr aplicaciones casi sin límite de capacidades.

Ethereum es una red peer-to-peer (p2p) de máquinas virtuales que cualquier desarrollador puede
utilizar para ejecutar aplicaciones distribuidas (dApps). Estos programas informáticos pueden ser
cualquier cosa, pero la red está optimizada para llevar a cabo reglas que se ejecutan mecánicamente
cuando se cumplen ciertas condiciones, como un contrato. Ethereum utiliza su propia blockchain
pública descentralizada para almacenar, ejecutar y proteger criptográficamente estos contratos. 
Cada computadora de su red descarga una pequeña máquina virtual para sincronizarse con la
blockchain de Ethereum y volver a estar disponible para ejecutar contratos. Esta red distribuida de
computadoras proporciona convenientemente la seguridad, confiabilidad y potencia de computación
necesarias para llevar a cabo los arreglos diseñados. Por supuesto, esta red de consenso no es
gratuita ni privada, por lo que los desarrolladores sólo la utilizan para llegar a un consenso
sobre los resultados y cuando sus datos pueden ser públicos.

La intención de Ethereum es crear un protocolo alternativo para la construcción de dApps,
proporcionando un conjunto diferente de compensaciones que se cree serán muy útiles para una gran
clase de aplicaciones descentralizadas, con especial énfasis en situaciones en las que el tiempo de
desarrollo rápido, la seguridad para aplicaciones pequeñas y poco utilizadas, y la capacidad de las
 diferentes aplicaciones para interactuar muy eficientemente, son importantes.
Ethereum hace esto construyendo lo que es esencialmente la última capa fundacional abstracta: una
blockchain con un lenguaje de programación Turing completo integrado, que permite a cualquiera
escribir contratos inteligentes y aplicaciones descentralizadas donde puede crear sus propias
reglas arbitrarias para la propiedad, formatos de transacción y funciones de transición de estado.


\subsection{Fíat Currency (dinero fiduciario)}
Comúnmente se dice que el bitcoin no está respaldado por nada, y eso es cierto. Las monedas fiat
modernas tampoco están respaldadas por nada, sin embargo, son diferentes: endosados
por un gobierno, una moneda fiduciaria es mantenida por defecto por cualquiera que pague impuestos
y compre bonos del Estado. Algunas ventas internacionales de materias primas están denominadas en
dólares, también (por ejemplo, el petróleo) dando a la gente otra razón para retener dólares.

En el caso de las criptomonedas, persisten los problemas de adopción. Hoy en día, estos tokens
digitales siguen siendo una capa de pago público, rápido y seguro en la parte superior del sistema
de dinero fiduciario existente; un despliegue experimental que podría algún día crecer para
reemplazar los pagos centralizados utilizadas hoy en día por empresas como Visa y MasterCard.

Se vislumbran grandes posibilidades en el horizonte, ya que los gobiernos y el sector privado s
encuentran en una situación difícil. Los inversores institucionales empiezan a crear grandes
mercados para los productos y servicios financieros denominadas en criptomonedas. Los bancos
centrales pueden incluso adoptar esta tecnología. Al menos un país ha emitido un dólar digital
utilizando el software Bitcoin: Barbados. Otros están investigando activamente el prospecto.


\subsubsection{Ether}
Hoy en día, Bitcoin (denotado por el símbolo BTC) es utilizado por personas, gobiernos y empresas
para transferir valor y comprar productos o servicios. Cada vez que envían bitcoins, pagan una
pequeña cuota a la red, que está denominada en bitcoins. Ether, denotado por el símbolo ETH, se
puede utilizar de forma similar. Para entender el camino a seguir, se necesita saber algunas cosas.

Primero, el ether tiene otro uso: puede pagar para ejecutar programas en la red de Ethereum.
Estos programas pueden transferir el ether ahora, o en el futuro, o cuando se cumplan ciertas
condiciones (especificadas en los contratos).
Debido a su capacidad de pagar por la ejecución de transacciones en el futuro, el ether puede
también ser considerado un producto básico, como el combustible para que la red ejecute
aplicaciones y servicios. Por lo tanto, tiene una dimensión adicional de valor intrínseco sobre los
bitcoins; no es sólo una reserva de valor.

Hoy en día, el abrumador uso de las monedas fiduciarias podría sugerir que
las criptomonedas son "peor dinero", es decir, más propensas a la inutilidad a largo plazo.
Y sin embargo, los bitcoins y el ether son famosamente acaparados por los poseedores, e incluso
mantenidos en un fideicomiso por al menos una compañía hasta el momento de escribir esto:
Grayscale, una subsidiaria de Digital Currency Group es un buen ejemplo.
Mientras tanto, los bancos centrales de Occidente experimentan con tipos de interés cercanos a cero
y flexibilización cuantitativa, también conocida como impresión de dinero, en una situación cada
vez más peligrosa y desesperada por mantener la inflación y la deflación bajo control.

Con la recompensa de bitcoin reduciéndose a la mitad cada cuatro años, la política monetaria
mundial se ve afectada. En general la incertidumbre económica, y la disminución de la confianza en
las monedas fiduciarias, enormes sumas de dinero latente en criptomonedas "acaparadas" están siendo
arrastradas al mercado por los precios más altos del servicio de demanda genuina. Esto se refleja
en los precios cada vez mayores de la mayoría de los tokens criptográficos, por muy volátiles que
sean sus precios intradía. Este acto de equilibrio entre acaparadores, especuladores, y
"derrochadores" crea un mercado próspero y saludable para la criptomoneda, y sugiere que
los criptokens como una clase de activos ya están sirviendo a los propósitos del dinero, y mucho
más.


\subsection{Protocolos}

\subsection{Smart Contracts}

\subsection{Solidity programming language}

\subsection{Base de datos distribuida, ¿Dónde están los datos?}
