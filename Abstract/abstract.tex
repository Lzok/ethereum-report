% ************************** Thesis Abstract *****************************
% Use `abstract' as an option in the document class to print only the titlepage and the abstract.
\begin{abstract}

Ethereum representa la segunda generación de tecnología blockchain al proporcionar
una plataforma informática abierta y global que permite el intercambio de criptomonedas
(Ether) y el desarrollo de aplicaciones de contratos inteligentes (smart contracts) autoverificables.
Los contratos inteligentes presentan una base para poseer activos digitales y una variedad de
aplicaciones descentralizadas dentro del área de la blockchain. Ethereum y los contratos inteligentes
son públicos, distribuidos e inmutables.

El objetivo de este estudio es definir el concepto de Blockchain, analizar casos de uso de los
contratos inteligentes específicamente en la Blockchain Ethereum y examinar tanto cuestiones generales
como específicas acerca de cómo ésta realiza todo el trabajo.

Uno de los puntos importantes de esta investigación es entender el funcionamiento de toda la
blockchain Ethereum dado su exponencial crecimiento en el último tiempo y poder desarrollar una
pequeña aplicación con Solidity, Node.js y React.js que sea capaz de compilar smart contracts escritos en Solidity,
deployarlos a la red privada de pruebas Rinkeby e interactuar con este contrato deployado tanto trayendo como enviándole información.

La aplicación a desarrollar será una plataforma de crowdfunding donde los usuarios que aporten 
a los proyectos, además deberán aprobar solicitudes de pagos creadas por el owner del proyecto.
Sin éstas, el dinero del proyecto no podrá ser movido de la plataforma. Esto surge debido a que
en la historia de kickstarter por dar un ejemplo (https://www.kickstarter.com/) ha habido casos
en donde proyectos con mucha financiación y popularidad han desaparecido de la plataforma junto con
el dinero recaudado provocando fraudes.

\end{abstract}
