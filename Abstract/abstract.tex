% ************************** Thesis Abstract *****************************
% Use `abstract' as an option in the document class to print only the titlepage and the abstract.
\begin{abstract}

Ethereum representa la segunda generación de tecnología blockchain al proporcionar
una plataforma informática abierta y global que permite el intercambio de criptomonedas
(Ether) y el desarrollo de aplicaciones de contratos inteligentes (smart contracts) autoverificables.
Los contratos inteligentes presentan una base para poseer activos digitales y una variedad de
aplicaciones descentralizadas dentro del área de la blockchain. Ethereum y los contratos inteligentes
son públicos, distribuidos e inmutables.

El objetivo de este estudio es definir el concepto de Blockchain, analizar casos de uso de los
contratos inteligentes específicamente en la Blockchain Ethereum y examinar tanto cuestiones generales
como específicas acerca de cómo ésta realiza todo el trabajo.

Uno de los puntos importantes que se destacarán a lo largo de este documento serán los problemas de
escalabilidad que tiene esta tecnologia ya que, en principio, las criptomonedas no se diseñaron
inicialmente con la idea de un uso y una adaptación generalizados. A medida que el número de
transacciones diarias continúa aumentando, un número cada vez mayor de cuestiones están apareciendo.


\end{abstract}
